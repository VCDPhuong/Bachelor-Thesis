\documentclass[12pt,english,a4paper]{article}
\usepackage[a4paper, bindingoffset=0.2in, left=0.75in, right=0.75in, top=1in, bottom=1in, footskip=.25in]{geometry}
\usepackage{tikz-feynman}
\usetikzlibrary{shapes.geometric}
\usepackage{array}
\usepackage{comment}
\usepackage{animate}
\usepackage{amsmath}
\usepackage{cancel}
\usepackage{multicol}
\usepackage{physics}
\usepackage{graphicx}
\usepackage{amssymb}
\usepackage{hyperref}
\usepackage[utf8]{vietnam}

\linespread{1.25}
\begin{document}
	\begin{center}
		{\large {ĐẠI HỌC QUỐC GIA THÀNH PHỐ HỒ CHÍ MINH}\\ {TRƯỜNG ĐẠI HỌC KHOA HỌC TỰ NHIÊN}\\
			{ {KHOA VẬT LÝ – VẬT LÝ KỸ THUẬT}}}\\[1cm]
		{\textbf{THÔNG TIN KHÓA LUẬN TỐT NGHIỆP}}
	\end{center}
	\quad Tên khóa luận: Tính Toán Phổ Hấp Thụ Tuyến Tính Của  $\mathrm{MoS}_2$\\\null
	\quad Ngành: Vật Lý Học\\\null
	\quad Chuyên ngành: Vật Lý Lý Thuyết\\\null
	\quad Họ tên sinh viên: Võ Châu Đức Phương\\\null
	\quad Mã số sinh viên: 20130008\\\null
	\quad Khóa đào tạo: 2020\\\null
	\quad Người hướng dẫn 1: TS. Huỳnh Thanh Đức \\\null
	\quad Người hướng dẫn 2:\\\null
	\subsection*{Tóm tắt nội dung khóa luận tốt nghiệp}
	\quad kim loại chuyển tiếp Dichalcogenides (TMD) là chất bán dẫn đầy hứa hẹn do tính chất điện tử và quang điện tử phi thường của chúng. Trong cấu trúc đơn lớp, tương tác Coulomb không bị chắn làm gia tăng năng lượng liên kết exciton. Do đó, mô hình xấp xỉ electron độc lập không đáng tin cậy để mô phỏng các thí nghiệm, đặc biệt là trong các hệ thấp chiều. Trong khóa luận này, chúng tôi sử dụng Hamiltonian ba dải liên kết chặt để tính toán cấu trúc dải và phương trình Bloch bán dẫn để mô hình hóa phản ứng của mật độ electron trong vật liệu với laser theo thời gian để tính toán phổ hấp thụ tuyến tính và trích xuất năng lượng liên kết exciton. Kết quả tính toán về năng lượng liên kết exciton dựa trên mô hình này phù hợp thực nghiệm và được chứng minh là chính xác hơn một số lý thuyết trước đó, dự đoán năng lượng liên kết exciton quá lớn. Từ kết quả này, chúng ta có thể sử dụng nó để tính toán các hiệu ứng tuyến tính và phi tuyến khác bị ảnh hưởng bởi năng lượng liên kết exciton như dòng quang điện, tạo sóng hài bậc cao và tạo dải bên bậc cao. Để cải thiện kết quả trong tương lai, chúng tôi có thể tăng mật độ của điểm k để có kết quả hội tụ tốt hơn. Đồng thời tính đến tương tác chắn khi tăng mật độ electron trên các dải dẫn khi vượt quá giới hạn kích thích thấp để tính toán phổ hấp thụ đầy đủ.\\\null
	\\[1cm]
	\textbf{Từ khóa:} năng lượng liên kết exciton, phổ hấp thụ tuyến tính, kim loại chuyển tiếp Dichalcogenides, phương trình Bloch bán dẫn, quang - điện tử
	\newpage
	\begin{center}
		{\large {VIETNAM NATIONAL UNIVERSITY OF HO CHI MINH CITY}\\ {UNIVERSITY OF SCIENCE}\\
		{ {FACULTY OF PHYSICS AND ENGINEERING PHYSICS}}}\\[1cm]
		{\textbf{GRADUATION THESIS INFORMATION}}
	\end{center}
	\quad Thesis title: Calculation of the Linear-Absorption Spectrum of $\mathrm{MoS}_2$\\\null
	\quad Major: Physics\\\null
	\quad Specialty: Theoretical Physics\\\null
	\quad Name of Student: Vo Chau Duc Phuong\\\null
	\quad Student ID: 20130008\\\null
	\quad Academic Year: 2020\\\null
	\quad Supervisor: Dr. Huynh Thanh Duc \\\null
	\subsection*{Thesis Abstract}
	\quad Transition metal dichalcogenides (TMD) are promising semiconductors due to their extraordinary electronic and optoelectronic properties. Before calculating the nonlinear optical properties of this material, we need to calculate the linear absorption spectrum. In monolayer TMD, the Coulomb interaction heavily influences the increase in exciton binding energies. Therefore, the independent electron approximation model is unreliable for simulating the experiment, especially in low-dimensional systems. In this work, we use a minimum three-band tight-binding Hamiltonian for the band structure calculation and the semiconductor Bloch equations to model the response of electron density in material to a laser over time to calculate the linear absorption spectra (LAS) and extract the exciton binding energy from it. Our calculation on exciton binding energy (0.25 eV) based on this model proves to agree with the experiment. It proves to be more accurate than some previous theories, which predicted exciton binding energy too large (0.5-1 eV). From this work, we can use it to calculate other linear- and nonlinear-effects that are affected by exciton binding energy such as photo-current, high harmonic generation, and high-order sideband generation. For further work, we can increase the density of the k-point for better convergence results. Take into account the shield coulomb interaction when increasing the density of electrons on the conduction bands when excess of the low excitation limit for full absorption spectra.\\\null
	\\[1cm]
	\textbf{Keywords:} exciton binding energy, linear absorption spectrum, Semiconductor Bloch Equation Transition metal dichalcogenides, optoelectronic
\end{document}